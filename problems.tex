\documentclass{article}
\usepackage[centertags]{amsmath}
\usepackage{amsfonts}

\newtheorem{problem}{Problem}

\begin{document}

\title{UCI Math Department's Intro to MATLab/Python--Problem Set}
\maketitle

With each of the following problems, see if you can minimize compute time, lines of code, or both!

\begin{problem}
Create a function to test if a number is prime. Then, write a script which takes in user input and tells the user if the input is prime or not.
\end{problem}

\begin{problem}
Consider the famous Lotka-Volterra equations used in predator-prey models:

\begin{align*}
\frac{dx}{dt} & = \alpha x - \beta xy \\
\frac{dy}{dt} & = \delta xy - \gamma y
\end{align*}

Create a phase diagram for positive parameter values by plotting several trajectories in the $xy$-plane with distinct initial conditions.

\end{problem}

\begin{problem}

Write a function that takes in a Markov Transition Matrix and outputs the absorbing states.
Make sure that your function checks that the input actually is a Markov Matrix!
Test your code by creating a Markov Transition Matrix and iterating $N$ times and see if you get convergence to the absorbing state(s).

\end{problem}

\begin{problem}

Choose three points to form an equilateral triangle, then a point in the interior at random.  Choose a vertex at random and select a second point, halfway between the previous point and the chosen vertex.  Repeat this many times and plot the results.

\end{problem}


\end{problem}

\begin{problem}

Using any scheme you like, solve the differential equation y'' = -y, y(0) = y(2pi) = 1 numerically on 0 <= x <= 2pi.  Plot your results and compare your answer to the one found in the file FEMExample.m.

\end{problem}

\begin{problem}

Model a store with 3 cashiers that check out one item per time unit, where customers show up at times defined by an exponential random variable with a mean of 10 time units.  Each customer should have a Poisson distributed random number of items with a mean of 30.  Customers join open lines randomly, or if each line is busy, they choose randomly with weights proportional to the number of items in OTHER lines. (That is, if there are people with 10 total items waiting in line 1, 20 in line 2 and 30 in line 3, a new customer has a 5/12 chance of joining line 1, 4/12 of joining line 2 and a 3/12 chance of joining line 3.)  Plot your results as a histogram of the amount of time each line has a given number of customers.

\end{problem}

\end{document}
